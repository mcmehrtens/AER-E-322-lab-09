% AER E 361 Mission Report Template
% Spring 2023
% Template created by Yiqi Liang and Professor Matthew Nelson

% Document Configuration DO NOT CHANGE
\documentclass[12 pt]{article}
% --------------------LaTeX Packages---------------------------------
% The following are packages that are used in this report.
% DO NOT CHANGE ANY OF THE FOLLOWING OR YOUR REPORT WILL NOT COMPILE
% -------------------------------------------------------------------

\usepackage{hyperref}
\usepackage{parskip}
\usepackage{titlesec}
\usepackage{titling}
\usepackage{graphicx}
\usepackage{graphviz}
\usepackage[T1]{fontenc}
\usepackage{titlesec, blindtext, color} %for LessIsMore style
\usepackage{tcolorbox} %for references box
\usepackage[hmargin=1in,vmargin=1in]{geometry} % use 1 inch margins
\usepackage{float}
\usepackage{tikz}
\usepackage{svg} % Allows for SVG Vector graphics
\usepackage{textcomp, gensymb} %for degree symbol
\hypersetup{
	colorlinks=true,
	linkcolor=blue,
	urlcolor=cyan,
}
\usepackage{biblatex}
\addbibresource{lab-report-bib.bib}
\usepackage{amsmath}
\usepackage{listings}
\usepackage{multicol}
\usepackage{array}

\usepackage{hologo} %KYR: for \BibTeX
%\usepackage{algpseudocode}
%\usepackage{algorithm}
% This configures items for code listings in the document
\usepackage{xcolor}

\usepackage{fancyhdr} % Headers/Footers
\usepackage{siunitx} % SI units
\usepackage{csquotes} % Display Quote
\usepackage{microtype} % Better line breaks

\definecolor{commentsColor}{rgb}{0.497495, 0.497587, 0.497464}
\definecolor{keywordsColor}{rgb}{0.000000, 0.000000, 0.635294}
\definecolor{stringColor}{rgb}{0.558215, 0.000000, 0.135316}
\definecolor{mygreen}{rgb}{0,0.6,0}
\definecolor{mygray}{rgb}{0.5,0.5,0.5}
\definecolor{mymauve}{rgb}{0.58,0,0.82}

\lstdefinestyle{customc}{
  belowcaptionskip=1\baselineskip,
  breaklines=true,
  frame=L,
  xleftmargin=\parindent,
  language=C,
  showstringspaces=false,
  basicstyle=\footnotesize\ttfamily,
  keywordstyle=\bfseries\color{green!40!black},
  commentstyle=\itshape\color{purple!40!black},
  identifierstyle=\color{blue},
  stringstyle=\color{orange},
 }

 \lstset{ %
  backgroundcolor=\color{white},   % choose the background color; you must add \usepackage{color} or \usepackage{xcolor}
  basicstyle=\footnotesize,        % the size of the fonts that are used for the code
  breakatwhitespace=false,         % sets if automatic breaks should only happen at whitespace
  breaklines=true,                 % sets automatic line breaking
  captionpos=b,                    % sets the caption-position to bottom
  commentstyle=\color{commentsColor}\textit,    % comment style
  deletekeywords={...},            % if you want to delete keywords from the given language
  escapeinside={\%*}{*)},          % if you want to add LaTeX within your code
  extendedchars=true,              % lets you use non-ASCII characters; for 8-bits encodings only, does not work with UTF-8
  frame=tb,	                   	   % adds a frame around the code
  keepspaces=true,                 % keeps spaces in text, useful for keeping indentation of code (possibly needs columns=flexible)
  keywordstyle=\color{keywordsColor}\bfseries,       % keyword style
  language=Python,                 % the language of the code (can be overrided per snippet)
  otherkeywords={*,...},           % if you want to add more keywords to the set
  numbers=left,                    % where to put the line-numbers; possible values are (none, left, right)
  numbersep=8pt,                   % how far the line-numbers are from the code
  numberstyle=\tiny\color{commentsColor}, % the style that is used for the line-numbers
  rulecolor=\color{black},         % if not set, the frame-color may be changed on line-breaks within not-black text (e.g. comments (green here))
  showspaces=false,                % show spaces everywhere adding particular underscores; it overrides 'showstringspaces'
  showstringspaces=false,          % underline spaces within strings only
  showtabs=false,                  % show tabs within strings adding particular underscores
  stepnumber=1,                    % the step between two line-numbers. If it's 1, each line will be numbered
  stringstyle=\color{stringColor}, % string literal style
  tabsize=2,	                   % sets default tabsize to 2 spaces
  title=\lstname,                  % show the filename of files included with \lstinputlisting; also try caption instead of title
  columns=fixed                    % Using fixed column width (for e.g. nice alignment)
}

\lstdefinestyle{customasm}{
  belowcaptionskip=1\baselineskip,
  frame=L,
  xleftmargin=\parindent,
  language=[x86masm]Assembler,
  basicstyle=\footnotesize\ttfamily,
  commentstyle=\itshape\color{purple!40!black},
}

\lstset{escapechar=@,style=customc}

\titlelabel{\thetitle.\quad}

% From here on out you can start editing your document
\newcommand{\subtitle}[1]{%
  \posttitle{%
    \par\end{center}
    \begin{center}\LARGE#1\end{center}
    \vskip0.5em}%
}

\newcommand{\etal}{\textit{et al}., }
\newcommand{\ie}{\textit{i}.\textit{e}., }
\newcommand{\eg}{\textit{e}.\textit{g}., }

% Define the headers and footers
\setlength{\headheight}{70.63135pt}
\geometry{head=70.63135pt, includehead=true, includefoot=true}
\pagestyle{fancy}
\fancyhead{}\fancyfoot{} % clears the headers/footers
\fancyhead[L]{\textbf{AER E 322}}
\fancyhead[C]{\textbf{Aerospace Structures Pre-Laboratory}\\
			  \textbf{Lab 9 Introduction to Nondestructive Evaluation}\\
			  Section 4 Group 2\\
			  Matthew Mehrtens\\
			  \today}
\fancyhead[R]{\textbf{Spring 2023}}
\fancyfoot[C]{\thepage}

\begin{document}
\section*{Question 1}
(\num{4} points) \textit{In a sentence or two, state your definition of NDE.}

\textbf{Solution:}

Nondestructive Evaluation (NDE) is a term describing analysis techniques that do not significantly disturb the material being tested. NDE techniques utilize everything from chemical testing to ultrasonic or acoustic evaluation.

\section*{Question 2}
(\num{3} points each) \textit{Under what technology category (e.g. visual examination, chemical testing, etc.) is each of the three NDE techniques learned in this lab?}

\textbf{Solution:}

\begin{itemize}
	\item Tap Testing; tap testing is a form of mechanical vibration testing.
	\item Ultrasonic Testing; ultrasonic testing is a form of mechanical vibration testing.
	\item Eddy Current Testing; eddy current testing is a form of electromagnetic radiation testing.
\end{itemize}

\section*{Question 3}
(\num{5} points) \textit{How does tap testing determine the local stiffness of the part under testing?} (\num{3} points) \textit{What is being measured?}

\textbf{Solution:}

The computer-aided tap tester (CATT) taps the material and measures the contact time of the tap. By using the relationship
\begin{align}
	T&=\pi\sqrt{\frac{M}{K}}
\end{align}
where $T$ is the contact time in seconds, $M$ is the mass of the tapper head, and $K$ is the local stiffness of the material. The CATT is used in a grid pattern to generate a test image.

\section*{Question 4}
(\num{3} points each) \textit{What are the three basic components in an ultrasonic immersion testing system?}

\textbf{Solution:}

\begin{enumerate}
	\item Transducer (TR)
	\item Pulser/Receiver (P/R)
	\item Oscilloscope
\end{enumerate}

\section*{Question 5}
(\num{15} points) \textit{For the ultrasonic immersion testing, you are instructed to follow a strategy of ``sequential search''. Describe what this is and how it will be executed in the lab.}

\textbf{Solution:}

The sequential search method describes a method of locating flaws by first identifying a set of intermediate targets. In this lab, we will locate the signal just leaving the transducer face, then the front surface, back surface, and finally the flaw.

To accomplish this, we first have to ensure the water path is within \qty{4.5}{in}. This is done by using the oscilloscope and the equations in the lecture notes. This process will also find the signal of the transducer face and the front surface. Next, we need to locate the back surface signal. Using the known thickness of the disk, we can calculate the time difference between the front and bottom surfaces.

Now that the back surface has been located, our group will search spatially, \ie search all possible areas on the back surface, until the flaw is located. Finally, we have to determine the flaw size using amplitude profiling. This is done by assuming the peak-to-peak amplitude occurs at the at flaw center.

\section*{Question 6}
(\num{10} points each) \textit{Describe what are A-, B- and C-scans in an ultrasonic testing?}

\textbf{Solution:}

The A-scan is a one-dimensional graph of the UT signal amplitudes. The height of the peaks shows the strength of the reflection and the distance along the $x$-axis shows the depth of the reflections.

The B-scan creates a type of contour graph. The graph is generated by plotting the amplitude values of each A-scan as a vertical stack of pixels to create a 3-D contour.

The C-scan creates a 2-D graph of the top-down view of a specimen. The plot is generated by plotting each peak-peak amplitude of the A-scans as a colored pixel value to generate a 2-D image.

\section*{Question 7}
(\num{10} points) \textit{From what you learn in lecture notes, how does eddy current testing work?}

\textbf{Solution:}

Eddy current testing utilizes electromagnetic probes that apply an alternating electric current to a substance. This electricity creates eddy currents (EC) which are disturbed by flaws in the material.

There are two types of probes: an absolute probe which uses a single coil and a differential probe which uses two coil elements. Using either probe, it's important to zero out the probes frequently by taking measurements on a ``defect-free'' sample.

The test works due to the skin effect, \ie the effect that causes eddy current density to decrease deeper in a material. The EC penetration depth can be calculated with the equation
\begin{align}
	\delta&\approx\frac{1}{\sqrt{\pi{}f\mu\sigma}}
\end{align}
where $\delta$ is the penetration depth, $f$ is the frequency, $\mu$ is the magnetic permeability of the material, and $\sigma$ is the electrical conductivity of the material.

\section*{Question 8}
(\num{10} points) \textit{Why is eddy current testing useful for sorting out metallic materials?}

\textbf{Solution:}

The absolute EC sensor is especially good at measuring defects in metallic material because its very sensitive to electrical conductivity and magnetic conductivity. Since metallic materials are generally great conductors, this makes EC tests especially effective. The material properties of metallic materials allow the EC sensors to detect very fine or minute defects.
\end{document}
